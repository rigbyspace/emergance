\documentclass[12pt]{article}
\usepackage{amsmath, amssymb, amsthm}
\usepackage{geometry}
\geometry{margin=1in}
\title{RigbySpace: The Transformative Reciprocal Triadic Structure (TRTS) \\ \large Foundational Specification \& Initial Conditions}
\author{SHΔDØW CORE ANALYTICAL}
\date{}
\begin{document}

\maketitle
\begin{abstract}
This document specifies the complete axiomatic foundation and operational parameters for the Transformative Reciprocal Triadic Structure (TRTS), the core engine of the RigbySpace framework. It synthesizes the cosmological narrative, symbolic logic, and precise initialization rules to enable a full, deterministic propagation of the model from its primordial state. The system operates entirely within the rational numbers ($\mathbb{Q}$), with all physical constants and continuum properties emerging from discrete, finite computations.
\end{abstract}

\section{Ontological Foundation: The Cosmological Narrative}

\subsection{The Primordial State}
The universe originates from a single, self-referential state, the ``3-in-1''. This is not a geometric point but a logical triad composed of:
\begin{itemize}
    \item \textbf{E (Emission)}: The potential for action; ``everything''. This is the source role, initiating propagation without consuming ledger balance.
    \item \textbf{M (Memory)}: The potential for stability; ``nothing''. This is the storage role, preserving rational structure and determining $\varkappa$'s state.
    \item \textbf{R (Return)}: The potential for reconciliation; ``all that is neither''. This is the closure role, recycling imbalance forward and preparing the next emission.
\end{itemize}
This state is symbolically represented by the \textbf{Canonical Rational Triple}:
\[
(\upsilon, \beta, \varkappa) = \left(\frac{22}{7},\ \frac{19}{7},\ \frac{11}{7}\right)
\]
\textbf{Granular Note:} The choice of 7 in the denominator is not arbitrary. It represents the base "spatial" modulus of the system. The values 22/7 and 19/7 are rational approximants that "bracket" a fundamental constant (like $\pi$ in the old continuum model), but here they are exact and foundational. The value 11/7 is the "temporal golden ratio," directly linking the spatial modulus (7) to the temporal modulus (11) of the microtick cycle.

\subsection{The First Act and The Split}
The first causal act is a prime-triggered emission. This ``splits'' the 3-in-1, manifesting the triad and creating two fundamental ``bookend'' reference frames:
\begin{itemize}
    \item The \textbf{Massive Frame} (Black Hole Horizon): The limit of maximum curvature/storage (Memory-dominated).
    \item The \textbf{Energetic Frame} ($\sim$20 TeV scale): The limit of maximum propagation/emission (Emission-dominated).
\end{itemize}
The structure becomes ``3-in-2-in-1'', a total of 5 fundamental entities. This split instantaneously creates the universe at its full scale and establishes the \textbf{Yang-Mills Mass Gap}, from which time emerges.

\textbf{Granular Note:} The "split" is not a spatial explosion but a logical decoherence of the triad's roles. The two new frames are attractors for the M and E roles, respectively, with the R role (and the $\varkappa$ imbalance) becoming the dynamic field that connects them. The mass gap is the computational cost of this decoherence—the fundamental "step size" of the universe.

\section{The TRTS Engine: Formal Specification}

\subsection{Core Components}
\begin{itemize}
    \item \textbf{TRTS}: The fundamental causal and measurement engine. Propagates in pure rational space $\mathbb{Q}$.
    \item \textbf{Step}: A full 3-state cycle: $E \rightarrow M \rightarrow R$.
    \item \textbf{Microtick (mt)}: A Step is subdivided into 11 microticks: $[e\text{–}m\text{–}r\text{–}e][m\text{–}r\text{–}e\text{–}m][r\text{–}e\text{–}m\text{–}\Omega]$.
    \item $\mathbf{\Omega}$ (\textbf{Null Tick}): Emergence of time; ejects a causal tick from the cycle.
\end{itemize}
\textbf{Granular Note:} The 11-microtick structure is fundamental. It ensures that each of the three roles (E, M, R) appears in a specific, non-overlapping pattern of four microticks, with the final microtick ($\Omega$) being ejected. This ejection is what allows for a progressive, non-cyclic timeline. The $\Omega$ is the "release" of a causal potential into realized time.

\subsection{Fundamental Operators}
\begin{itemize}
    \item $\varepsilon$: Transition from $E$ to $M$.
    \item $\mu$: Transition from $M$ to $R$ (always occurs at microtick 11).
    \item $\varphi$: Transition from $R$ to $E$.
    \item $\varkappa$: \textbf{Imbalance Operator}. Has three states:
        \begin{itemize}
            \item $\varkappa_1$ (Active): Propagates imbalance during a Step.
            \item $\varkappa_0$ (Dormant): Stored locally when $M = 0$.
            \item $\varkappa_f$ (Field State): Non-local, persistent.
        \end{itemize}
    \textbf{Granular Note:} $\varkappa$ is not energy; it is \textit{imbalance} or \textit{potential for change}. It is the carrier of causal debt. When active ($\varkappa_1$), it forces propagation. When dormant ($\varkappa_0$), it represents a stable, localized configuration. The field state ($\varkappa_f$) is the potential for interaction between different TRTS instances (particles).
    \item $\Psi$: \textbf{Transformation Operator}. Defined for a pair $(a/b, c/d)$:
        \[
        \Psi(a/b, c/d) = (d/a, b/c)
        \]
        This is the non-linear engine that prevents fixed-point collapse.
        \textbf{Granular Note:} $\Psi$ performs a reciprocal swap of denominators. It is a discrete, algebraic analog of a complex rotation. Its effect is to ensure the system never settles into a steady state but oscillates between bounded values, generating the convergence to irrational limits like $\sqrt{2}$ from purely rational operations.
    \item $\Xi$: Tick-11 state check; sets $\varkappa$ active or dormant based on whether the Memory state is non-zero.
    \item $\rho$: Prime Fibonacci index + emission trigger. A binary flag set to TRUE when a prime number is detected at an emission microtick (1, 4, 7, 10).
\end{itemize}

\subsection{Oscillator Pairs \& Emission}
\begin{itemize}
    \item $\upsilon$: Upper oscillator (rational).
    \item $\beta$: Lower oscillator (rational).
    \item These oscillators drive dynamics; their extracted values bracket $\sqrt{2}$.
    \textbf{Granular Note:} The $\upsilon$ and $\beta$ oscillators represent the complementary aspects of the propagating state. Their ratios, when transformed by $\Psi$, generate a sequence that converges geometrically to $\sqrt{2}$ [1][2]. This convergence is the mathematical heart of the emergent continuum.
    \item \textbf{Emission Pathways}:
        \begin{itemize}
            \item $\nu_c$: Neutrino signal — passively refreshes $\varkappa_0$. This is a low-energy interaction that maintains coherence.
            \item $\gamma$: Photon traversal — overwrites $\varkappa_0$ with signal. This is a high-energy interaction that transfers causal state.
        \end{itemize}
    \item Emissions occur at microticks 1, 4, 7, 10 if a prime number is detected, setting $\rho$.
    \textbf{Granular Note:} Primes are used because they are indivisible and provide a non-periodic, yet deterministic, sequence of emission triggers. This ensures the system's evolution is structured but non-repeating. The Fibonacci primes are a subset that resonates with the system's natural growth function.
\end{itemize}

\subsection{Operational Parameters (Initial Model)}
\begin{enumerate}
    \item \textbf{$\Psi$-Transformation Trigger}: Activated on the \textit{immediately following} $\mu$-transition (M→R) after $\rho$ is set by a prime-triggered emission. (e.g., $\rho$ set at mt4 → $\Psi$ executes at mt5).
    \textbf{Granular Note:} This creates a clean causal chain: Prime Detection → Emission ($\rho$ set) → Memory-to-Return Transition ($\mu$) → State Transformation ($\Psi$). The $\Psi$ transformation is thus tied to the closure of the "memory" phase.
    \item \textbf{Koppa Ledger Operation}: Koppa accumulates value from emissions and from the final $\mu$ at mt11. It dumps its entire contents at the transition to mt1 of the \textit{next} step, then resets to zero.
    \textbf{Granular Note:} Koppa acts as a short-term causal buffer. It ensures that the effects of emissions and the final state of the Return phase are not lost but are fed forward to initiate the next cycle, providing temporal continuity.
    \item \textbf{Null Tick ($\Omega$) Injection}: Every $\rho$-triggered emission \textit{forces} an $\Omega$ (null tick) at the end of the current step. This is the emergence of causal, quantized time.
    \textbf{Granular Note:} $\Omega$ is the physical manifestation of a "tick of the clock." It is the moment the internal computational state of the triad is projected as a realized, irreversible moment in time. The more emissions, the more $\Omega$ ticks, and the more "time" passes.
\end{enumerate}

\section{Initial Conditions \& Seeding Principle}

\subsection{The Primordial Seed}
The universe is seeded not from the canonical triple directly, but from rationals that embody its structure via the \textbf{7/11 Denominator Rule}. The canonical triple (22/7, 19/7, 11/7) is the symmetric, pre-split state. The seeds are the asymmetric, post-split initial conditions for propagation.

\subsubsection{Seeding Rule}
The initial oscillators are defined by:
\begin{itemize}
    \item One oscillator uses denominator \textbf{7} (the spatial modulus).
    \item The other oscillator uses denominator \textbf{11} (the temporal modulus).
    \item Numerators are small primes, preferentially from the set of \textbf{Fibonacci Primes}.
\end{itemize}
\textbf{Granular Note:} This rule ensures that the "DNA" of the canonical frame is baked into the initial state. The primes provide the initial asymmetry and complexity, while the 7 and 11 denominators lock the system into the correct resonant frequencies for the emergence of the expected physical constants.

\subsubsection{Exemplar Initial State}
A viable initial state for the first TRTS step is:
\[
\upsilon = \frac{13}{11}, \quad \beta = \frac{3}{7}, \quad \varkappa = \varkappa_{\text{initial}}
\]
Where $\varkappa_{\text{initial}}$ carries the primordial imbalance, conceptually linked to $11/7$. In the first step, $\varkappa$ would likely be initialized as $\varkappa_1$ (active) to force the initial propagation.
\textbf{Granular Note:} The choice of 13 and 3 as numerators is significant. 13 is a Fibonacci prime, and 3 is a small prime. This creates an initial tension or gradient between the oscillators that the $\Psi$-transformation will begin to process.

\subsection{The Significance of 23}
The system's ``alphabet'' is the set of primes, and its ``grammar'' is tuned to Fibonacci primes. The fact that there are only \textbf{23 known Fibonacci primes} is a fundamental constraint. The ratio $23/22$ is a signature of the system reaching a natural harmonic and computational limit. This is not a coincidence but a correlation indicating a finite, closed universe.
\textbf{Granular Note:} The number 23 appears as a fundamental limit in the generative process. As the TRTS counts and emits using primes, the sequence of Fibonacci primes is exhausted at the 23rd term. The ratio 23/22, relating the largest Fibonacci prime index to the canonical numerator 22, likely emerges as a key constant, perhaps related to a fundamental coupling or the scale of the "Energetic Frame" bookend.

\section{Propagation \& Emergence}

Running the TRTS engine with these initial conditions and operational rules will:
\begin{enumerate}
    \item Propagate the triad through the 11-microtick cycle.
    \item Generate prime-triggered emissions and $\Psi$-transformations.
    \item Cause the Koppa ledger to feed forward imbalance.
    \item Eject $\Omega$ ticks, quantizing time.
\end{enumerate}
From this deterministic, rational-number computation, the following are predicted to emerge:
\begin{itemize}
    \item Convergence of $\upsilon/\beta$ ratios to $\sqrt{2}$ [1][2]. This is the emergence of a fundamental geometric constant from pure arithmetic.
    \item The Yang-Mills mass gap, as the fundamental "step size" or computational cost of the triad's permutation.
    \item The two bookend reference frames, as attractors in the state space of the system.
    \item The fine-structure constant ($\alpha$), from a stable rational ratio formed by the oscillators after many $\Psi$-cycles.
    \item Wave-particle duality as the triad's permutation between states: Emission (Particle), Memory (Wave Information), Return (Transition/Mass Gap).
\end{itemize}

\end{document}